\documentclass[a4paper,oneside,10pt,DIV12,headsepline,footexclude,headexclude]{scrartcl}


%% Normal LaTeX or pdfLaTeX? %%%%%%%%%%%%%%%%%%%%%%%%%%%%%%%%
%% ==> The new if-Command "\ifpdf" will be used at some
%% ==> places to ensure the compatibility between
%% ==> LaTeX and pdfLaTeX.
\newif\ifpdf
\ifx\pdfoutput\undefined
	\pdffalse              %%normal LaTeX is executed
\else
	\pdfoutput=1
	\pdftrue               %%pdfLaTeX is executed
\fi

%% Packages for Graphics & Figures %%%%%%%%%%%%%%%%%%%%%%%%%%
\ifpdf %%Inclusion of graphics via \includegraphics{file}
	\usepackage[pdftex]{graphicx} %%graphics in pdfLaTeX
\else
	\usepackage[dvips]{graphicx} %%graphics and normal LaTeX
\fi
\graphicspath{{fig/}}

%% Fonts for pdfLaTeX %%%%%%%%%%%%%%%%%%%%%%%%%%%%%%%%%%%%%%%
%% ==> Only needed, if cm-super-fonts are not installed
%%\ifpdf
%	%\usepackage{ae}       %%Use only just one of these packages:
%	%\usepackage{zefonts}  %%depends on your installation.
%%\else
%	%%Normal LaTeX - no special packages for fonts required
%%\fi

\renewcommand{\rmdefault}{pbk} % bookman
\renewcommand{\sfdefault}{phv} % helvetica (avantgarde = pag)
\renewcommand{\ttdefault}{pcr} % courier
\renewcommand{\familydefault}{phv}

%\usepackage{cmbright}  % computer modern bright - not for pdf


\areaset{16cm}{24cm}
\addtolength{\topskip}{0.5cm}


% texttt hyphenation
\newcommand{\origttfamily}{}
\let\origttfamily=\ttfamily
\renewcommand{\ttfamily}{\origttfamily \hyphenchar\font=`\-}


\let\ifpdf\relax

\usepackage[T1]{fontenc}
\usepackage[latin1]{inputenc}
\usepackage{array}
\usepackage{float}
%\usepackage{paralist}
\usepackage{color}
\usepackage{colortbl}

\usepackage{listings}
\lstset{language=Java,basicstyle=\ttfamily\small,tabsize=2}


%% Line Spacing %%%%%%%%%%%%%%%%%%%%%%%%%%%%%%%%%%%%%%%%%%%%%
%\usepackage{setspace}
%\singlespacing        %% 1-spacing (default)
%\onehalfspacing       %% 1,5-spacing
%\doublespacing        %% 2-spacing

\linespread{1.05}
\addtolength{\parskip}{0.175\baselineskip}

\widowpenalty = 10000
\clubpenalty = 10000


%%%%%%%%%%%%%%%%%%%%%%%%%%%%%%%%%%%%%%%%%%%%%%%%%%%%%%%%%%%%%
%% DOCUMENT
%%%%%%%%%%%%%%%%%%%%%%%%%%%%%%%%%%%%%%%%%%%%%%%%%%%%%%%%%%%%%
\begin{document}

%% File Extensions of Graphics %%%%%%%%%%%%%%%%%%%%%%%%%%%%%%
%% ==> This enables you to omit the file extension of a graphic.
%% ==> "\includegraphics{title.eps}" becomes "\includegraphics{title}".
%% ==> If you create 2 graphics with same content (but different file types)
%% ==> "title.eps" and "title.pdf", only the file processable by
%% ==> your compiler will be used.
%% ==> pdfLaTeX uses "title.pdf". LaTeX uses "title.eps".
\ifpdf
	\DeclareGraphicsExtensions{.pdf,.jpg,.png}
\else
	\DeclareGraphicsExtensions{.eps}
\fi


\pagestyle{plain} %Now display headings: headings / fancy / ...

\title{\Large Extended Abstract on Mobile Robotics}

\author{\large Constantin Schieber, 1228774, Technische Universit�t Wien}
%\date{} %%If commented, the current date is used.

\maketitle

\begin{section}{Introduction and problem statement}

Detecting and tracking people on a mobile robotic platform with real time
requirements and hardware constraints is an upcomming issue.
Knowing the positions of people over time is important for many applications that
focus on human - robot interaction, e.g. autonomously following a person.

The following work summarizes five papers on this topic to introduce the 
available hardware / algorithms and its utilization for people tracking.

\end{section}

\begin{section}{Short summary of the Papers to Investigate}

\begin{subsection}{OpenPTrack}
The first paper ~\cite{munaro2014openptrack} introduces OpenPTrack. OpenPTrack
is a framework built with the Robotic Operating System (ROS) and enables the user
to utilize and calibrate a network of multiple RGB-D cameras. Detection of people
happens in a distributed fashion while the tracking is done in a single node that
processes all detections from the network.
\end{subsection}

\begin{subsection}{Person Tracking and Following with 2D Laser Scanners}
In this paper ~\cite{7139259} an approach that utilizes depth sensors - such as
laser or RGB-D - for sensing its surroundings is introduced. The implementation
is provided as open source ROS package which can be used with any depth sensing
based hardware on a height of 30cm. A retraining of the human-confidence 
learning algorithm may be necessary when different sensor resolutions are used.\\

Detection and tracking works with moving and non-moving persons in cluttered 
enviroments. It is independet of light conditions, works in close proximity
to the tracked person and does not need an \textit{a priori} occupancy grid map 
of the surroundings.

%The work is subsequently splitted into three sub-problems that are (1) detection,
%(2) tracking and (3) following.
%It moving and
%non-moving targets in cluttered enviroments without requiring an \texit{a priori}
%occupancy grid map of the enviroment.
%
%Measuerement takes place on a height of 30cm, which means that mostly the legs
%are tracked. The points returned from the distance measurement of the laser are
%then assigned to clusters that will. Assigning is based on a treshhold that will
%put a persons legs most of the time in two distinct clusters.
%Clusters will then be classified as human and non-human on basis of geometric
%features. 
%
%Laser sensing is light invariant and has a wide wide field of view - making it
%suitable for tracking in close proximity to a person and outdoor tasks.
%
%
%The research is open source and provided as a ROS package.
 
\end{subsection}

\begin{subsection}{Computationally Intelligent System for Thermal Vision People
Detection and Tracking in Robotic Applications}
This work ~\cite{ciric2013computationally} uses thermal vision and ultrasonic
sensors. 
It uses the distinct thermal profile of people to detect and track them. This is
independet of light conditions.
Aquiring the position of a human in an thermal image is realized by segmentating
the image in Regions Of Interests (ROIs). Data of the thermal sensor 
represents differences in the thermal energy of objects in an enviroment.
Thresholding this data then allows to extract segments of interest. 
As most thermal image segmentation algorithms fail to provide real time perfomance
a new algorithm is introduced.\\
Utilizing fuzzy membership functions that determine how strongly a RGB color 
belongs to the fore- or background it is possible to have a fast and reliable
segmentation algorithm. Applying such fuzzy logic on RGB color instead of greyscale
images provides much better results. A disadvantage is that this method won't yield
good results in situations where more precise tracking is required, e.g. 
for body joints.\\
A neuro-fuzzy classifier then determines if a segment is human or not.\\
This approach provides fast and reliable tracking of one person.
It is able to detect and track multiple persons but - depending on the interactions
between these people - loses track easily.


\end{subsection}


\begin{subsection}{4}
\end{subsection}

\begin{subsection}{5}
\end{subsection}

\end{section}


\bibliography{references}
\bibliographystyle{IEEEtran}

\end{document}
