\documentclass[a4paper,oneside,10pt,DIV12,headsepline,footexclude,headexclude]{scrartcl}


%% Normal LaTeX or pdfLaTeX? %%%%%%%%%%%%%%%%%%%%%%%%%%%%%%%%
%% ==> The new if-Command "\ifpdf" will be used at some
%% ==> places to ensure the compatibility between
%% ==> LaTeX and pdfLaTeX.
\newif\ifpdf
\ifx\pdfoutput\undefined
	\pdffalse              %%normal LaTeX is executed
\else
	\pdfoutput=1
	\pdftrue               %%pdfLaTeX is executed
\fi

%% Packages for Graphics & Figures %%%%%%%%%%%%%%%%%%%%%%%%%%
\ifpdf %%Inclusion of graphics via \includegraphics{file}
	\usepackage[pdftex]{graphicx} %%graphics in pdfLaTeX
\else
	\usepackage[dvips]{graphicx} %%graphics and normal LaTeX
\fi
\graphicspath{{fig/}}

%% Fonts for pdfLaTeX %%%%%%%%%%%%%%%%%%%%%%%%%%%%%%%%%%%%%%%
%% ==> Only needed, if cm-super-fonts are not installed
%%\ifpdf
%	%\usepackage{ae}       %%Use only just one of these packages:
%	%\usepackage{zefonts}  %%depends on your installation.
%%\else
%	%%Normal LaTeX - no special packages for fonts required
%%\fi

\renewcommand{\rmdefault}{pbk} % bookman
\renewcommand{\sfdefault}{phv} % helvetica (avantgarde = pag)
\renewcommand{\ttdefault}{pcr} % courier
\renewcommand{\familydefault}{phv}

%\usepackage{cmbright}  % computer modern bright - not for pdf


\areaset{16cm}{24cm}
\addtolength{\topskip}{0.5cm}


% texttt hyphenation
\newcommand{\origttfamily}{}
\let\origttfamily=\ttfamily
\renewcommand{\ttfamily}{\origttfamily \hyphenchar\font=`\-}


\let\ifpdf\relax

\usepackage[T1]{fontenc}
\usepackage[latin1]{inputenc}
\usepackage{array}
\usepackage{float}
%\usepackage{paralist}
\usepackage{color}
\usepackage{colortbl}

\usepackage{listings}
\lstset{language=Java,basicstyle=\ttfamily\small,tabsize=2}


%% Line Spacing %%%%%%%%%%%%%%%%%%%%%%%%%%%%%%%%%%%%%%%%%%%%%
%\usepackage{setspace}
%\singlespacing        %% 1-spacing (default)
%\onehalfspacing       %% 1,5-spacing
%\doublespacing        %% 2-spacing

\linespread{1.05}
\addtolength{\parskip}{0.175\baselineskip}

\widowpenalty = 10000
\clubpenalty = 10000


%%%%%%%%%%%%%%%%%%%%%%%%%%%%%%%%%%%%%%%%%%%%%%%%%%%%%%%%%%%%%
%% DOCUMENT
%%%%%%%%%%%%%%%%%%%%%%%%%%%%%%%%%%%%%%%%%%%%%%%%%%%%%%%%%%%%%
\begin{document}

%% File Extensions of Graphics %%%%%%%%%%%%%%%%%%%%%%%%%%%%%%
%% ==> This enables you to omit the file extension of a graphic.
%% ==> "\includegraphics{title.eps}" becomes "\includegraphics{title}".
%% ==> If you create 2 graphics with same content (but different file types)
%% ==> "title.eps" and "title.pdf", only the file processable by
%% ==> your compiler will be used.
%% ==> pdfLaTeX uses "title.pdf". LaTeX uses "title.eps".
\ifpdf
	\DeclareGraphicsExtensions{.pdf,.jpg,.png}
\else
	\DeclareGraphicsExtensions{.eps}
\fi


\pagestyle{plain} %Now display headings: headings / fancy / ...

\title{\Large Extended Abstract on Mobile Robotics}

\author{\large Constantin Schieber, 1228774, Technische Universitaet Wien}
%\date{} %%If commented, the current date is used.

\maketitle

\begin{section}{Introduction and problem statement}

Detecting and tracking people on a mobile robotic platform with real time
requirements and hardware constraints is an upcoming issue.
Knowing the positions of people over time is important for many applications that
focus on human - robot interaction, e.g. autonomously following a person.

The following work summarizes five papers on this topic to introduce the 
available hardware / algorithms and its utilization for people tracking.

\end{section}

\begin{section}{Short summary of the Papers to Investigate}

\paragraph{OpenPTrack}\mbox{} \\
The first paper ~\cite{munaro2014openptrack} introduces OpenPTrack. OpenPTrack
is a framework built with the Robotic Operating System (ROS) and enables the user
to utilize and calibrate a network of multiple RGB-D cameras. Detection of people
happens in a distributed fashion while the tracking is done in a single node that
processes all detection from the network.\\
The relative position of the cameras to each other can be easily calibrated using 
a checkerboard and real-time visual feedback from the ROS application.\\
Real-time tracking perfomance of groups is achieved under most light conditions
due to the usage of different sensor readings (depth, infrared, color). 

\paragraph{Person Tracking and Following with 2D Laser Scanners}
\mbox{} \\
In this paper ~\cite{7139259} an approach that utilizes depth sensors - such as
laser or RGB-D - for sensing its surroundings is introduced. The implementation
is provided as open source ROS package which can be used with any depth sensing
based hardware on a height of 30cm. A retraining of the human-confidence 
learning algorithm may be necessary when different sensor resolutions are used.\\

Detection and tracking works with moving and non-moving persons in cluttered 
environments. It is independent of light conditions, works in close proximity
to the tracked person and does not need an \textit{a priori} occupancy grid map 
of the surroundings. Depending on the used laser noise from the sun can be a 
problem.
A joint leg tracker is used instead of tracking individual legs.

%The work is subsequently splitted into three sub-problems that are (1) detection,
%(2) tracking and (3) following.
%It moving and
%non-moving targets in cluttered enviroments without requiring an \texit{a priori}
%occupancy grid map of the enviroment.
%
%Measuerement takes place on a height of 30cm, which means that mostly the legs
%are tracked. The points returned from the distance measurement of the laser are
%then assigned to clusters that will. Assigning is based on a treshhold that will
%put a persons legs most of the time in two distinct clusters.
%Clusters will then be classified as human and non-human on basis of geometric
%features. 
%
%Laser sensing is light invariant and has a wide wide field of view - making it
%suitable for tracking in close proximity to a person and outdoor tasks.
%
%
%The research is open source and provided as a ROS package.
 

\paragraph{Thermal vision based approach}
\mbox{} \\
This work ~\cite{ciric2013computationally} uses thermal vision and ultrasonic
sensors. 
It uses the distinct thermal profile of people to detect and track them. This is
independent of light conditions.
Acquiring the position of a human in an thermal image is realized by segmenting
the image in Regions Of Interests (ROIs). Data of the thermal sensor 
represents differences in the thermal energy of objects in an environment.
Thresholding this data then allows to extract segments of interest. 
As most thermal image segmentation algorithms fail to provide real time performance
a new algorithm is introduced.\\
Utilizing fuzzy membership functions that determine how strongly a RGB color 
belongs to the fore- or background it is possible to have a fast and reliable
segmentation algorithm. Applying such fuzzy logic on RGB color instead of greyscale
images provides much better results. A disadvantage is that this method won't yield
good results in situations where more precise tracking is required, e.g. 
for body joints.\\
A neuro-fuzzy classifier then determines if a segment is human or not.\\
This approach provides fast and reliable tracking of one person.
It is able to detect and track multiple persons but - depending on the interactions
between these people - loses track easily.



\paragraph{Ultrasonic Sensor based approach}
\mbox{} \\
The authors of this paper ~\cite{7064565} propose an ultrasonic/RF sensor 
based approach for single person tracking scenarios.
This system works well under any light conditions and provides good performance
in scenarios with terrain variations. 
ROS nodes are used for measurement and control tasks. A drawback is that the 
tracked person needs to carry an active sonar emitter.
\\
The sensor array on the robot is passive. A RF signal is sent to the robot,
indicating the start of emitting ultrasonic pressure waves from the tracked person.
The robot then measures the time of flight (ToF) and calculates the position
of the person from this data. Due to high noise, non-linear behavior and the
constant delay of the RF transmission triangulation based on this data is not
sufficient for people tracking. A Gaussian Process Regression (GPR) model of the
sensors is the novel approach of the authors to overcome this problem.
This system is then trained in a controlled setup to determine the likelihood
of measuring a given sonar signal at a certain location. A MATLAB-ROS Bridge is
used for this training.\\
Experiments show that this work is improving person tracking with ultrasonic/RF sensors
in a significant way, making it usable for systematic person tracking in a
outdoor setting with varying terrain.


\paragraph{Human robot spatial interaction}
\mbox{} \\
With the advance of robust people tracking implementations another aspect is 
becoming more important: Socially acceptable behavior of robot movement. That
means to not only treat humans as dynamic obstacles but to navigate around them
in a "natural" way.\\
This paper ~\cite{dondrup2015tracker} attempts to model such interactions and 
provides a processing pipeline implemented with ROS. It uses a 
combination of upper body and leg detection. Tracking and predicting movement of
people is achieved with a probabilistic real-time tracking framework.
Qualitative Spatial Relations (QSR) are derived from the movement of tracked
persons. This is done with Qualitative Trajectory Calculus (QTC). 
QTC state chains can be created and classified using a Hidden Markovchain Model (HMM)
that was set up in advance. For a distinct behavior a distinct HMM is trained.\\
It is shown that various QTC-based scenarios (e.g. overtaking, pass by) are 
significantly different from each other.\\

\end{section}


\bibliography{references}
\bibliographystyle{IEEEtran}

\end{document}
